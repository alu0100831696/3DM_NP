\documentclass[10pt, mathserif, profesionalfont]{beamer}

\usepackage[utf8]{inputenc}
\usepackage[spanish]{babel}



\usepackage{amsmath,amsfonts,amssymb,amsthm}
\usepackage{booktabs}
\usepackage{hyperref}
\usepackage{authblk}

\usepackage[acronym]{glossaries}

\renewcommand\Affilfont{\itshape\small}
\renewcommand\Authand{ y }

\newtheorem{thm}{Teorema}

\usepackage{array}
\newcolumntype{L}[1]{>{\raggedright\let\newline\\\arraybackslash\hspace{0pt}}p{#1}}
\newcolumntype{C}[1]{>{\centering\let\newline\\\arraybackslash\hspace{0pt}}p{#1}}
\newcolumntype{R}[1]{>{\raggedleft\let\newline\\\arraybackslash\hspace{0pt}}p{#1}}

\newcolumntype{X}[1]{>{\raggedright\let\newline\\\arraybackslash\hspace{0pt}}m{#1}}
\newcolumntype{Y}[1]{>{\centering\let\newline\\\arraybackslash\hspace{0pt}}m{#1}}
\newcolumntype{Z}[1]{>{\raggedleft\let\newline\\\arraybackslash\hspace{0pt}}m{#1}}

\usepackage{color}
\makeglossaries

\newacronym{ndtm}{NDTM}{Máquina de Turing No Determinista}

\newacronym{sat}{SAT}{SATISFACTIBILIDAD}

\newacronym{3dm}{3DM}{3-DIMENSIONAL MATCHING}

\usetheme{Boadilla}
\title{\Huge Demostraci\'{o}n de $\mathcal{NP}$-completitud \LaTeX \\3 DIMENSIONAL MATCHING}
\author{Sofía Pizarro Arbelo \and alu0100831696}
\institute[ULL]{Universidad de La Laguna}
\date{Curso 2017/2018}
\subject{Complejidad Computacional}

\AtBeginSubsection[]
{
  \begin{frame}<beamer>{Outline}
    \tableofcontents[currentsection,currentsubsection]
  \end{frame}
}

\begin{document}

\begin{frame}
  \titlepage
\end{frame}

\section{Introducción}
\begin{frame}{NP-completos}
    
\begin{block}
{\small 
\noindent ¿Son tratables los NP-completos?

\noindent Los NP-completos parecen intratables, aunque nadie ha sabido demostrar que los NP-completos son
intratables. Son todos equivalentes, es decir:
}
\end{block}

\begin{block}
{\small
\noindent Si se encuentra un algoritmo eficiente para un NP-completo entonces tenemos un algoritmo
eficiente para cualquiera de ellos.

\noindent Si probamos que un NP-completo no tiene algoritmos eficientes entonces ninguno los tiene.
}
\end{block}


\end{frame}

\section{Transformaciones polinomiales}
\begin{frame}{Transformaciones polinomiales I}
    
\begin{block}  
 Una transformación o reducción polinomial de un problema de decisión $\Pi$1 a uno $\Pi$2 es una función que se computa en tiempo polinomial y transforma una instancia I1 de $\Pi$1 en una instancia I2 de $\Pi$2 tal que I1 tiene respuesta "sí" para $\Pi$1 si y solo si I2 tiene respuesta "sí" para $\Pi$2.  

\end{block}

\end{frame}

\begin{frame}{Transformaciones polinomiales II}
    
\begin{block}  
El problema de decisión $\Pi$1 se reduce polinomialmente a otro problema de decisión $\Pi$2, $\Pi$1 $\leq$p $\Pi$2, si existe una transformación polinomial de $\Pi$1 a $\Pi$2. 

\end{block}
\begin{block}
Si $\Pi$'' $\leq$p $\Pi$' y $\Pi$' $\leq$p $\Pi$ entonces $\Pi$'' $\leq$p $\Pi$, ya que la composición de dos reducciones polinomiales es una reducción polinomial.
\end{block}  

\end{frame}

\section{Problemas $\mathcal{NP}$ Completos}
\begin{frame}{Problemas $\mathcal{NP}$ Completos}
    
\begin{block}  
Un problema $\Pi$ es $\mathcal{NP}$-completo si:
\begin{enumerate}

	\item $\Pi$ $\in$ $\mathcal{NP}$.
	\item Para todo $\Pi$' $\in$ $\mathcal{NP}$ , $\Pi$' $\leq$p $\Pi$.
	\end{enumerate}
		\includegraphics[width =0.5\textwidth]{1.jpg}

\end{block}


\end{frame}

\section{Descripción del problema 3DM}
\begin{frame}{Descripción del problema 3DM I}
    \begin{block}  
	Instancia
		\begin{itemize}
			\item Un conjunto M c W x X x Y	
			\begin{itemize}
				\item W $\cap$ Y $\cap$ X = $\phi$ (disjuntos)
				\item $\mid$W$\mid$ = $\mid$X$\mid$ = $\mid$Y$\mid$ = q
			\end{itemize}
		\end{itemize}
\end{block}


\end{frame}
\begin{frame}{Descripción del problema 3DM II}		
	 \begin{block}
	 Pregunta: L1 $\leq$n L
		\begin{itemize}
			\item $\mid$M'$\mid$ = q
			\item Todos los elementos W u X u Y están en alguna terceta de M' sin repetir ninguno.
		\end{itemize}
		
			\includegraphics[width =0.5\textwidth]{2.jpg}
\end{block}
\end{frame}

\section{Demostrar que un 3DM es NP completo}
\begin{frame}{3SAT $\alpha$ 3DM I}
    \includegraphics[width =0.5\textwidth]{3.jpg}

\end{frame}

\begin{frame}{3SAT $\alpha$ 3DM II}
	Notación
    \includegraphics[width =0.5\textwidth]{4.jpg}

\end{frame}
\begin{frame}{3SAT $\alpha$ 3DM III}
Tercetas de asignación
    \includegraphics[width =0.5\textwidth]{5.jpg}

\end{frame}
\begin{frame}{3SAT $\alpha$ 3DM IV}
M' será un matching con m elementos de Ti
    \includegraphics[width =0.5\textwidth]{6.jpg}

\end{frame}

\begin{frame}{3SAT $\alpha$ 3DM V}
Tercetas de satisfacción
    \includegraphics[width =0.5\textwidth]{7.jpg}

\end{frame}
\begin{frame}{3SAT $\alpha$ 3DM V}
Tercetas de relleno
    \includegraphics[width =0.5\textwidth]{8.jpg}

\end{frame}

\begin{frame}{3SAT $\alpha$ 3DM VI}
    \includegraphics[width =0.5\textwidth]{9.jpg}
    
\end{frame}
\begin{frame}{3SAT $\alpha$ 3DM VI}
   \includegraphics[width =0.5\textwidth]{10.jpg}
 \includegraphics[width =0.5\textwidth]{11.jpg}
\\\textit{\textbf{3-Dimensional Matching es $\mathcal{NP}$-COMPLETO}}

\end{frame}


\end{document}